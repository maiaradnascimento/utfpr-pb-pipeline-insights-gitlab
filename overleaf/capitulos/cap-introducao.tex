%%%% CAPÍTULO 1 - INTRODUÇÃO
%%
%% Deve apresentar uma visão global da pesquisa, incluindo: breve histórico, importância e justificativa da escolha do tema,
%% delimitações do assunto, formulação de hipóteses e objetivos da pesquisa e estrutura do trabalho.

%% Título e rótulo de capítulo (rótulos não devem conter caracteres especiais, acentuados ou cedilha)
\chapter{Introdução}\label{cap:introducao}

\textit{Pipelines} de Integração Contínua e Entrega Contínua (\gls{cicd}) são centrais na cultura \gls{devops}, mas gargalos, instabilidades e retrabalho ainda são frequentes em ambientes de larga escala. Em contextos com políticas restritivas de acesso (\textit{compliance}), é desejável extrair inteligência apenas dos artefatos de execução (metadados de \textit{pipelines}/\textit{jobs} e \textit{logs}) para orientar melhorias. Este trabalho propõe e valida um sistema que coleta, analisa e explica o comportamento de \textit{pipelines} no \gls{gitlab}, sem acesso ao código-fonte, para detectar anomalias e gerar recomendações aplicáveis.

\section{Problema e diferencial}\label{sec:problemaDiferencial}

O problema central deste trabalho consiste em identificar e priorizar, de forma automática, anomalias e oportunidades de otimização em \textit{pipelines} \gls{gitlab}, com baixo acoplamento e alto respeito à privacidade. O diferencial da proposta é operar apenas com dados de execução (\gls{api} + \textit{logs}) e entregar recomendações acionáveis, mantendo neutralidade em relação ao conteúdo dos repositórios e aos arquivos \texttt{.gitlab-ci.yml}.

\section{Objetivos}\label{sec:objetivos}

A seguir são apresentados os objetivos geral e específicos que o presente projeto pretende alcançar ao final de seu desenvolvimento.

\subsection{Objetivo geral}\label{subsec:objetivoGeral}

Desenvolver e validar um sistema baseado em análise estatística descritiva (incluindo métodos robustos como percentis) e \gls{ml} não supervisionado para detectar anomalias e gerar recomendações de otimização em \textit{pipelines} \gls{cicd} do \gls{gitlab}, operando sem acesso ao código-fonte.

\subsection{Objetivos específicos}\label{subsec:objetivosEspecificos}

\begin{enumerate}
    \item Coletar e normalizar dados de \textit{pipelines}/\textit{jobs} e trechos de \textit{logs} via \gls{api} do \gls{gitlab}.
    \item Aplicar modelos de detecção de anomalias (como \gls{if}) e contextualização por agrupamento (como \gls{kmeans}), combinados a análise estatística descritiva (percentis robustos e z-score).
    \item Gerar recomendações automáticas (ex: \textit{cache}, paralelização, \textit{retries}, \textit{timeouts}, ordenação de \textit{jobs}), com justificativas quantitativas e \textit{links} para evidências.
    \item Disponibilizar um \textit{dashboard} \gls{html} auto-contido com visualizações e explicações acessíveis, adequado para revisão rápida por times \gls{devops}.
\end{enumerate}

\section{Justificativa}\label{sec:justificativa}

A justificativa para este trabalho pode ser dividida em três dimensões:

\begin{itemize}
    \item \textbf{Organizacional:} reduz esforço manual de \textit{troubleshooting}, padroniza heurísticas e distribui conhecimento.
    \item \textbf{Técnica:} preserva \textit{compliance} e segurança ao não ler \gls{yaml}/código; facilita adoção em ambientes multi-time e multi-repo.
    \item \textbf{Acadêmica:} demonstra a aplicabilidade de IA/\gls{ml} a um problema prático de \gls{devops} com restrição de dados, fomentando pesquisa em observabilidade de \textit{pipelines}.
\end{itemize}

\section{Questões de pesquisa}\label{sec:questoesPesquisa}

Este trabalho busca responder às seguintes questões:

\begin{itemize}
    \item \textit{Q1.} É possível detectar anomalias úteis usando somente dados de execução?
    \item \textit{Q2.} Regras baseadas em percentis adaptativos (p95/p99) complementam \gls{ml} para recomendações explicáveis?
    \item \textit{Q3.} Qual o custo estimado de coleta/armazenamento e o tempo de processamento por análise para uso cotidiano?
\end{itemize}

\section{Estrutura do trabalho}\label{sec:estruturaTrabalho}

Este trabalho está organizado em capítulos. O Capítulo 1 apresentou a introdução, os objetivos e a justificativa deste trabalho.

O Capítulo 2 apresenta o referencial teórico, abrangendo conceitos essenciais como \gls{devops}, \gls{cicd}, detecção de anomalias, \gls{xai} e trabalhos relacionados.

O Capítulo 3 descreve a proposta do sistema, detalhando a arquitetura proposta, os componentes principais, o fluxo de processamento e a estrutura do banco de dados.

O Capítulo 4 descreve os materiais e o método empregados, detalhando as variáveis, o procedimento experimental e as métricas de avaliação.

O Capítulo 5 expõe os resultados obtidos com a execução da metodologia, incluindo artefatos gerados, achados ilustrativos e a validação da proposta.

O Capítulo 6 apresenta as conclusões do trabalho e as perspectivas para pesquisas futuras.

Por fim, são listadas as referências bibliográficas que fundamentam este documento.
