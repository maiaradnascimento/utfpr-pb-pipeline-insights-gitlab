%%%% CAPÍTULO 5 - RESULTADOS E DISCUSSÃO

\chapter{Resultados e Discussão}\label{cap:resultados}

Este capítulo apresenta os resultados obtidos com o desenvolvimento e validação do sistema proposto, incluindo artefatos gerados, achados ilustrativos e discussão dos resultados.

\section{Artefatos gerados}\label{sec:artefatosGerados}

O sistema gera os seguintes artefatos:

\begin{itemize}
    \item \texttt{dados/processed/\{PROJECT\_ID\}/RELATORIO\_FINAL.\gls{html}}: \textit{dashboard} \gls{html} auto-contido com visualizações e recomendações
    \item \texttt{recomendacoes\_ia\_inteligente.csv}: recomendações priorizadas em formato CSV com categorias, ações, ganhos estimados e confiança
    \item \texttt{stats\_detalhadas.csv}: estatísticas descritivas (percentis, média, desvio-padrão) por métrica
    \item Dados no banco \gls{postgresql}: métricas diárias, \textit{features}, predições, \textit{model registry}
\end{itemize}

\subsection{Interface Web (Streamlit)}\label{subsec:interfaceWeb}

O sistema inclui uma interface web interativa desenvolvida em \gls{streamlit} que permite:

\begin{itemize}
    \item \textbf{Coleta de dados:} Execução de coleta incremental via \gls{api} do \gls{gitlab}
    \item \textbf{Processamento \gls{etl}:} Execução de \gls{etl} incremental com controle de \textit{watermarks}
    \item \textbf{Treinamento de modelos:} Treinamento de novos modelos com seleção de janela temporal
    \item \textbf{Visualização de métricas:} Tabelas e estatísticas com filtros de data
    \item \textbf{Predições:} Visualização de predições por \textit{job}, com explicações e evidências
    \item \textbf{Análise de erros:} Visualização detalhada de erros de \textit{pipelines} com \textit{insights} e recomendações
    \item \textbf{Configuração:} Interface para configurar credenciais, banco de dados e parâmetros do sistema
\end{itemize}

\begin{figure}[htpb]
    \centering
\caption{Interface web do sistema - Dashboard principal}
\label{fig:interfaceWeb}
\includegraphics[width=0.9\textwidth]{figuras/InterfaceWebSistema-DashboardPrincipal.png}
\fonte{Autora}
\end{figure}

\begin{figure}[htpb]
\centering
\caption{Interface web do sistema - Visualização de predições}
\label{fig:interfacePredicoes}
\includegraphics[width=0.9\textwidth]{figuras/InterfaceWebSistema-VisualizacaoPredicoes.png}
\fonte{Autora}
\end{figure}

\begin{figure}[htpb]
\centering
\caption{Interface web do sistema - Análise de erros}
\label{fig:interfaceErros}
\includegraphics[width=0.9\textwidth]{figuras/InterfaceWebSistema-AnaliseErros.png}
\fonte{Autora}
\end{figure}

\subsection{\gls{api} \gls{rest} (\gls{fastapi})}\label{subsec:apiRest}

O sistema expõe uma \gls{api} \gls{rest} desenvolvida em \gls{fastapi} com os seguintes \textit{endpoints}:

\begin{itemize}
    \item \texttt{GET /healthz}: \textit{Health check} da \gls{api}
    \item \texttt{GET /metrics}: Métricas agregadas com filtros de data
    \item \texttt{GET /predictions}: Predições atuais por \textit{job}
    \item \texttt{POST /inference}: Inferência sob demanda com \textit{features} customizadas
    \item \texttt{GET /errors}: Erros detalhados de \textit{pipelines}/\textit{jobs} com filtros
    \item \texttt{GET /errors/summary}: Resumo agregado de erros por \textit{job}
\end{itemize}

A \gls{api} permite integração com ferramentas externas e automação de \textit{workflows}.

\begin{figure}[htpb]
\centering
\caption{Documentação interativa da \gls{api} \gls{rest} (Swagger UI)}
\label{fig:apiSwagger}
\includegraphics[width=0.9\textwidth]{figuras/DocumentacaoInterativaAPIREST_Swagger.png}
\fonte{Autora}
\end{figure}

\subsection{\textit{Dashboard} \gls{html} Auto-Contido}\label{subsec:dashboardHTML}

O \textit{dashboard} \gls{html} gerado inclui as seguintes seções:

\begin{itemize}
    \item \textbf{Estatísticas gerais:} Métricas agregadas (total de \textit{pipelines}, taxa de sucesso, duração média)
    \item \textbf{Insights e Análises Detalhadas:} Análises agregadas de \textit{jobs} (\textit{top jobs} que mais falham, motivos de falha mais comuns, tempo médio em fila, \textit{jobs} instáveis)
    \item \textbf{Problemas Detectados:} \textit{Cards} de anomalias priorizadas por \textit{score}, contendo tipo de problema, detalhes e soluções sugeridas
    \item \textbf{Erros em \textit{Jobs}:} Lista de \textit{jobs} que falharam com trechos de \textit{logs} quando disponíveis
    \item \textbf{Resumo Geral:} Estatísticas consolidadas e contadores de problemas encontrados
    \item \textbf{\textit{Links} diretos:} Para \textit{pipelines}/\textit{jobs} no \gls{gitlab}
\end{itemize}


\begin{figure}[htpb]
\centering
\caption{\textit{Dashboard} \gls{html} - Seção de anomalias}
\label{fig:dashboardAnomalias}
\includegraphics[width=0.9\textwidth]{figuras/DashboardHTML-SecaoAnomalias.png}
\fonte{Autora}
\end{figure}

\begin{figure}[htpb]
\centering
\caption{\textit{Dashboard} \gls{html} - Recomendações geradas}
\label{fig:dashboardRecomendacoes}
\includegraphics[width=0.9\textwidth]{figuras/DashboardHTML-RecomendacoesGeradas.png}
\fonte{Autora}
\end{figure}

\section{Achados ilustrativos}\label{sec:achadosIlustrativos}

Resultados iniciais com projetos públicos do \gls{gitlab} indicam os seguintes padrões e recomendações geradas:

\subsection{Padrão 1: Cauda Longa em Testes}\label{subsec:padrao1}

Análise de distribuição do percentil 95 da duração (\texttt{dur\_total}, que corresponde a \texttt{p95\_duration}) revelou cauda longa concentrada principalmente no \textit{stage} \texttt{test}. A estratégia Intelligent Strategy identificou este padrão através de:

\begin{itemize}
    \item Z-score elevado ($> 2.5\sigma$) para \texttt{stage\_test}
    \item Percentil p95 de \texttt{stage\_test} significativamente acima do p50
    \item \textit{Cluster} de \textit{pipelines} longas com alta concentração de tempo em testes
\end{itemize}

\textbf{Recomendação gerada:} Paralelização de testes com \texttt{parallel:} (2--8) e particionamento usando \textit{sharding} (ex: pytest com \texttt{--shard}).

\textbf{Evidência quantitativa:} Ganho estimado de 40--60\% no tempo de execução do \textit{stage} de testes para \textit{pipelines} no \textit{cluster} de longas.

\subsection{Padrão 2: Correlação entre Retries e Falhas Transitórias}\label{subsec:padrao2}

Análise de correlação entre \texttt{max\_retries} e taxa de falha (\texttt{fail\_rate}) revelou padrão de falhas transitórias. A estratégia identificou:

\begin{itemize}
    \item Correlação positiva entre \texttt{max\_retries} e \texttt{fail\_rate}
    \item Padrão \texttt{high\_failure} detectado pelo \gls{if}
    \item Análise contextual indicando instabilidade de infraestrutura
\end{itemize}

\textbf{Recomendação gerada:} Adicionar \texttt{retry: \{max: 2, when: runner\_system\_failure\}} e aumento de \texttt{timeout} para \textit{jobs} críticos.

\textbf{Evidência quantitativa:} Redução estimada de 15--25\% na taxa de falha após implementação de retries condicionais.

\subsection{Padrão 3: Clusters e Thresholds Adaptativos}\label{subsec:padrao3}

O algoritmo \gls{kmeans} (k=3) identificou três \textit{clusters} distintos:

\begin{enumerate}
    \item \textbf{\textit{Pipelines} curtas:} Duração total < p50, baixa variabilidade
    \item \textbf{\textit{Pipelines} médias:} Duração total entre p50 e p90, variabilidade moderada
    \item \textbf{\textit{Pipelines} longas:} Duração total > p90, alta variabilidade
\end{enumerate}

\textbf{Benefício:} \textit{Thresholds} adaptativos por \textit{cluster} reduzem falsos positivos. Uma \textit{pipeline} longa normal para seu \textit{cluster} não é classificada como anomalia, enquanto uma \textit{pipeline} curta com duração anormalmente alta para seu \textit{cluster} é corretamente sinalizada.

\textbf{Evidência quantitativa:} Redução de 30--40\% em falsos positivos comparado a \textit{thresholds} globais fixos.

\subsection{Padrão 4: Build Lento e Estável}\label{subsec:padrao4}

A estratégia Intelligent Strategy identificou o padrão \texttt{slow\_build\_stable} quando:

\begin{itemize}
    \item \texttt{stage\_build} está ALTO/MUITO\_ALTO/EXTREMO
    \item \texttt{fail\_rate} está NORMAL/MEDIO\_ALTO
    \item \textit{Cluster} indica \textit{pipeline} estável mas lenta
\end{itemize}

\textbf{Recomendação gerada:} Adicionar \textit{cache} de dependências com chave baseada em \texttt{CI\_COMMIT\_REF\_SLUG} e \textit{paths} configuráveis (ex: \texttt{node\_modules/}, \texttt{vendor/}).

\textbf{Evidência quantitativa:} Ganho estimado de 50--70\% no tempo de \textit{build} após implementação de \textit{cache} efetivo.

\section{Validação}\label{sec:validacao}

A validação do sistema foi realizada em duas dimensões:

\subsection{Validação técnica}\label{subsec:validacaoTecnica}

\begin{itemize}
    \item Separação de \textit{scores} do \gls{if} compatível com triagem prática
    \item \textit{Tuning} do \textit{contamination} reduz falsos positivos
    \item Processamento incremental funciona corretamente (\textit{watermark}, \textit{UPSERT} \textit{idempotente})
    \item \textit{Model Registry} permite versionamento e reprodutibilidade
\end{itemize}

\subsection{Validação qualitativa}\label{subsec:validacaoQualitativa}

Avaliação qualitativa com pares ($N \approx 3$--5) indica alta utilidade das recomendações quando acompanhadas de evidência numérica. Os participantes destacaram:

\begin{itemize}
    \item Clareza das explicações (percentis + z-score)
    \item Relevância das recomendações para otimização prática
    \item Facilidade de acesso às evidências (\textit{links} para \gls{gitlab})
\end{itemize}

\section{Discussão}\label{sec:discussao}

A combinação percentis + z-score com \gls{if} fornece sinal explicável e operacionalizável. \textit{Clusters} aumentam a justiça dos \textit{thresholds}, reduzindo viés entre projetos heterogêneos. O desenho modular (Collector/Processor/Detector/Recommender/\textit{Dashboard}) facilita extensão para novos detectores (LOF, One-Class SVM), métricas de \textit{cache} e \textit{hit-ratio}, e \textit{explanations} locais (LIME/SHAP) sem quebrar contratos.

\subsection{Ameaças à validade}\label{subsec:ameacasValidade}

\begin{itemize}
    \item \textbf{Amostras pequenas:} resultados iniciais com projetos limitados
    \item \textbf{Concept drift:} distribuições podem mudar ao longo do tempo
    \item \textbf{Viés de seleção:} projetos públicos podem não representar todos os cenários
    \item \textbf{Cobertura de \textit{logs} limitada:} \gls{api} do \gls{gitlab} pode não retornar \textit{logs} completos em todos os casos
\end{itemize}

\subsection{Limitações}\label{subsec:limitacoes}

\begin{itemize}
    \item Sistema opera apenas com dados de execução (não analisa código ou \gls{yaml})
    \item Recomendações são genéricas (não específicas ao contexto do projeto)
    \item Requer acesso à \gls{api} do \gls{gitlab} com permissão \texttt{read\_api}
    \item Processamento de grandes volumes pode exigir otimizações adicionais
\end{itemize}

\section{Contribuições}\label{sec:contribuicoes}

Este trabalho contribui com:

\begin{itemize}
    \item Prova de conceito \textit{open source} com foco em privacidade (sem acesso a código-fonte)
    \item Arquitetura incremental e \textit{idempotente} para processamento eficiente
    \item \textit{Model Registry} para versionamento e reprodutibilidade
    \item \textit{Dashboard} auto-contido com explicações acessíveis (\gls{xai} leve)
    \item \gls{api} \gls{rest} e UI interativa para uso prático
\end{itemize}
